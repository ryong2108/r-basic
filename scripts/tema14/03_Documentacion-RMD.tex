% Options for packages loaded elsewhere
\PassOptionsToPackage{unicode}{hyperref}
\PassOptionsToPackage{hyphens}{url}
%
\documentclass[
]{article}
\usepackage{amsmath,amssymb}
\usepackage{lmodern}
\usepackage{ifxetex,ifluatex}
\ifnum 0\ifxetex 1\fi\ifluatex 1\fi=0 % if pdftex
  \usepackage[T1]{fontenc}
  \usepackage[utf8]{inputenc}
  \usepackage{textcomp} % provide euro and other symbols
\else % if luatex or xetex
  \usepackage{unicode-math}
  \defaultfontfeatures{Scale=MatchLowercase}
  \defaultfontfeatures[\rmfamily]{Ligatures=TeX,Scale=1}
\fi
% Use upquote if available, for straight quotes in verbatim environments
\IfFileExists{upquote.sty}{\usepackage{upquote}}{}
\IfFileExists{microtype.sty}{% use microtype if available
  \usepackage[]{microtype}
  \UseMicrotypeSet[protrusion]{basicmath} % disable protrusion for tt fonts
}{}
\makeatletter
\@ifundefined{KOMAClassName}{% if non-KOMA class
  \IfFileExists{parskip.sty}{%
    \usepackage{parskip}
  }{% else
    \setlength{\parindent}{0pt}
    \setlength{\parskip}{6pt plus 2pt minus 1pt}}
}{% if KOMA class
  \KOMAoptions{parskip=half}}
\makeatother
\usepackage{xcolor}
\IfFileExists{xurl.sty}{\usepackage{xurl}}{} % add URL line breaks if available
\IfFileExists{bookmark.sty}{\usepackage{bookmark}}{\usepackage{hyperref}}
\hypersetup{
  pdftitle={prueba de preambulo},
  pdfauthor={Raul Yong},
  hidelinks,
  pdfcreator={LaTeX via pandoc}}
\urlstyle{same} % disable monospaced font for URLs
\usepackage[margin=1in]{geometry}
\usepackage{longtable,booktabs,array}
\usepackage{calc} % for calculating minipage widths
% Correct order of tables after \paragraph or \subparagraph
\usepackage{etoolbox}
\makeatletter
\patchcmd\longtable{\par}{\if@noskipsec\mbox{}\fi\par}{}{}
\makeatother
% Allow footnotes in longtable head/foot
\IfFileExists{footnotehyper.sty}{\usepackage{footnotehyper}}{\usepackage{footnote}}
\makesavenoteenv{longtable}
\usepackage{graphicx}
\makeatletter
\def\maxwidth{\ifdim\Gin@nat@width>\linewidth\linewidth\else\Gin@nat@width\fi}
\def\maxheight{\ifdim\Gin@nat@height>\textheight\textheight\else\Gin@nat@height\fi}
\makeatother
% Scale images if necessary, so that they will not overflow the page
% margins by default, and it is still possible to overwrite the defaults
% using explicit options in \includegraphics[width, height, ...]{}
\setkeys{Gin}{width=\maxwidth,height=\maxheight,keepaspectratio}
% Set default figure placement to htbp
\makeatletter
\def\fps@figure{htbp}
\makeatother
\setlength{\emergencystretch}{3em} % prevent overfull lines
\providecommand{\tightlist}{%
  \setlength{\itemsep}{0pt}\setlength{\parskip}{0pt}}
\setcounter{secnumdepth}{-\maxdimen} % remove section numbering
\ifluatex
  \usepackage{selnolig}  % disable illegal ligatures
\fi

\title{prueba de preambulo}
\author{Raul Yong}
\date{20/11/2021}

\begin{document}
\maketitle

\hypertarget{titulo-1}{%
\section{Titulo 1}\label{titulo-1}}

\hypertarget{titulo-2}{%
\subsection{Titulo 2}\label{titulo-2}}

\hypertarget{tittulo-3}{%
\subsubsection{Tittulo 3}\label{tittulo-3}}

Ejemplo de \emph{palabras en cursiva }.

Ejemplo de \textbf{Palabras en Negrita}.

Ejemplo de palabras con anotaciones elevadas R\textsuperscript{@}.

Dale Click al siguiente link :
\textbf{\href{https://www.latinka.com.pe/p/\#}{Loteria la Tinka}}

endash : -- Y entonces dijo Juan Gabriel : \ldots{}

emdash : --- Como Juan queria explicar

Para graficos es :

\includegraphics{C:/USER/003_CURSOS/PROJECTS/MATEMATICAS/r-basic/teoria/Imgs/easy.png}

Ecuaciones \(S= \pi\cdot r^2\)

\begin{center}\rule{0.5\linewidth}{0.5pt}\end{center}

\begin{quote}
Hacemos otra cosa
\end{quote}

\hypertarget{lista-no-ordenada}{%
\subsubsection{Lista no ordenada}\label{lista-no-ordenada}}

\begin{itemize}
\tightlist
\item
  Item 1
\item
  Item 2

  \begin{itemize}
  \tightlist
  \item
    sub item2.1
  \item
    sub item2.2
  \end{itemize}
\item
  Item 3
\end{itemize}

\hypertarget{lista-ordenada}{%
\subsubsection{Lista Ordenada}\label{lista-ordenada}}

\begin{enumerate}
\def\labelenumi{\arabic{enumi}.}
\tightlist
\item
  Item 1
\item
  Item 2
\item
  Item 3

  \begin{itemize}
  \tightlist
  \item
    Sub item 3.1
  \item
    Sub item 3.4
  \end{itemize}
\item
  Item 4
\item
  Ultimo Item
\end{enumerate}

\hypertarget{tablas}{%
\subsubsection{Tablas}\label{tablas}}

\begin{longtable}[]{@{}lll@{}}
\toprule
Alumno & Notas & Edad \\
\midrule
\endhead
Juan & 10 & 20 \\
Raul & 15 & 25 \\
Carlos & 13 & 35 \\
\bottomrule
\end{longtable}

\end{document}
